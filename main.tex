%%%%%%%%%%%%%%%%%
% This is an sample CV template created using altacv.cls
% (v1.2, 11 February 2020) written by LianTze Lim (liantze@gmail.com). Now compiles with pdfLaTeX, XeLaTeX and LuaLaTeX.
%
%% It may be distributed and/or modified under the
%% conditions of the LaTeX Project Public License, either version 1.3
%% of this license or (at your option) any later version.bb
%% The latest version of this license is in
%%    http://www.latex-project.org/lppl.txt
%% and version 1.3 or later is part of all distributions of LaTeX
%% version 2003/12/01 or later.
%%%%%%%%%%%%%%%%

%% If you need to pass whatever options to xcolor
\PassOptionsToPackage{dvipsnames}{xcolor}

%% If you are using \orcid or academicons
%% icons, make sure you have the academicons
%% option here, and compile with XeLaTeX
%% or LuaLaTeX.
% \documentclass[10pt,a4paper,academicons]{altacv}

%% Use the "normalphoto" option if you want a normal photo instead of cropped to a circle
% \documentclass[10pt,a4paper,normalphoto]{altacv}

\documentclass[11pt,a4paper,ragged2e]{altacv}

%% AltaCV uses the fontawesome and academicon fonts
%% and packages.
%% See http://texdoc.net/pkg/fontawesome and http://texdoc.net/pkg/academicons for full list of symbols. You MUST compile with XeLaTeX or LuaLaTeX if you want to use academicons.

% Change the page layout if you need to
\geometry{left=0.7cm,right=0.7cm,top=0.7cm,bottom=0.7cm,columnsep=1.2cm}

% The paracol package lets you typeset columns of text in parallel
\usepackage{paracol}

% Change the font if you want to, depending on whether
% you're using pdflatex or xelatex/lualatex
\ifxetexorluatex
  % If using xelatex or lualatex:
  \setmainfont{Lato}
\else
  % If using pdflatex:
  \usepackage[utf8]{inputenc}
  \usepackage[T1]{fontenc}
  \usepackage[default]{lato}
\fi
\usepackage{hyperref}
% Change the colours if you want to
\definecolor{Mulberry}{HTML}{72243D}
\definecolor{SlateGrey}{HTML}{2E2E2E}
\definecolor{LightGrey}{HTML}{666666}
\definecolor{White}{HTML}{FFFFFF}
\definecolor{Black}{HTML}{000000}
\colorlet{heading}{Black}
\colorlet{accent}{Black}
\colorlet{emphasis}{Black}
\colorlet{body}{Black}

% Change the bullets for itemize and rating marker
% for \cvskill if you want to
\renewcommand{\itemmarker}{{\small\textbullet}}
\renewcommand{\ratingmarker}{\faCircle}

\begin{document}
\name{Harith Al-Safi}
\tagline{Electronics and Computer Engineer}
\location{Leeds, UK}
\email{\href{mailto:harith.alsafi@gmail.com}{harith.alsafi@gmail.com}}
\phone{\href{tel:+447444585915}{+447444585915}}
\homepage{\href{https://harith.io/}{https://harith.io/}}
\linkedin{\href{https://www.linkedin.com/in/harith-al-safi}{linkedin.com/in/harith-al-safi}}
\github{\href{https://www.github.com/harith-alsafi}{github.com/harith-alsafi}}

\makecvheader

%% Depending on your tastes, you may want to make fonts of itemize environments slightly smaller
% \AtBeginEnvironment{itemize}{\small}

%% Set the left/right column width ratio to 6:4.
\columnratio{0.5}

% Start a 2-column paracol. Both the left and right columns will automatically
% break across pages if things get too long.
\begin{paracol}{1}
\smallskip
\cvsection{\faBlackTie \, About me}

An engineer with great enthusiasm towards data science and Artificial Intelligence. Some of my hobbies are working out, trading and investment. A deterministic outlook characterize my approach to tackling challenges.\smallskip


\cvsection{\faBriefcase \, Experience}

\cvevent{Robot Engineer}{Robot fight league}{Dec 2023 - Mar 2024 (4 Months)}{Leeds, UK}
\begin{itemize}
\item Designed a 3D CAD for robot fight league using Solidworks and Blender and Jira for project management
\item Implemented a control system using Arduino and C++ for the robot's movement, sensors and pnuematic system
\item Built the robot with 3D printing, metal cutting and soldering for the electronics
\item Tested reinforcement learning algorithms using Python and KerasRL for the robot interactions
\item Managed a team of 5 engineers to develop the robot and compete in the league ending up 1\textsuperscript{st} place
\end{itemize}
\tightdivider

\cvevent{Frontend Developer}{Velz Travel}{Aug 2023 - Nov 2023 (4 Months)}{Remote}
\begin{itemize}
\item Directed the frontend of a travel web app for a startup built with Next.js, React.js and Typescript
\item Developed user interface using Material UI following Figma design guidelines and business logic using Nodejs
\item Collaborated with developers using Jira whilst working on Java Springboot backend with PostgreSQL and Docker
\item Conducted valuable tests of AI implementation using NLP and LLM's of the itinerary suggestion feature
\item Delivered app through CI/CD, agile development, automated testing with 93\% coverage and web deployment 
\end{itemize}
\tightdivider

\cvevent{Software Engineer (Placement Year)}{Johnson Controls R\&D department (JCI)}{July 2022 - July 2023 (12 Months)}{London UK}
\begin{itemize}
\item Administered C++ projects while fixing bugs, adding features, refactoring and migrating 90\% of codebase to C\#
\item Developed C\# .NET apps and libraries for IoT, fire detection simulations, R\&D, and data analysis
\item Participated in a tech challenge by designing analytics web dashboard with React JS for predictive maintenance 
\item Invented a custom scripting language with a compiler for C\# app increasing development productivity by 40\%
\item Managed Python automation scripts for automated builds of .NET applications and processing data  
\item Managed a team of 10 interns to create a business plan and ended up placing  2\textsuperscript{nd} amongst the whole company
\end{itemize}
\tightdivider

\cvevent{Machine Learning Engineer Freelance}{Paperound}{Dec 2021 - Mar 2022 (4 Months)}{Leeds UK}
% \cvevent{C++ Developer Freelance}{Paperound}{Dec 2021 - Mar 2022 (4 Months)}{Leeds UK}
% \cvevent{Software Developer Freelance}{Paperound}{Dec 2021 - Mar 2022 (4 Months)}{Leeds UK}
\begin{itemize}
\item Developed full stack web app using React.js, Next.js, MongoDB with user authentication and CRUD operations
\item Developed a cross platform mobile notes app using Flutter and Firebase alongside a recommendation system
\item Developed signal processing filters and modulators using SciPy Python and Fourier analysis for RF signals 
\item Applied data visualization using Python Matplotlib to present analytical insights to non-technical stakeholders
\item Developed natural language proccessing applications using LLM's and Python for sentiment analysis
\item Developed object detection and tracking using OpenCV and Python for detecting planets in telescope images
\item Utilized Python, TensorFlow, and Scikit-learn to build, train, and evaluate neural networks
\item Developed a reinforcement learning model using KerasRL for predicting stock prices
\item Developed Serverless REST API with C \# and Azure for IoT Sensors using ASP.NET 
\item Developed distributed system using C++ OpenMP, OpenCL and CUDA for parallel statistical computing
\item Developed low level ARM embedded systems with C whilst using memory allocation and register programming
\item Engineered a desktop QT application using C++ and Cmake for video analysis and processing
\item Engineered a high performance C++ linear algebra library for data analysis through Jenkins, Docker and Cmake
\end{itemize}
\tightdivider

\cvevent{Forex Day Trader}{ICMarkets FX}{Mar 2021 - Nov 2021 (9 Months)}{Leeds UK}
\begin{itemize}
\item Researched financial factors such as interest rates, GDP, inflation, and employment to predict market trends
\item Conducted quantitative research on statistical methods and their applications in financial markets.
\item Designed a trading strategy using technical, fundamental analysis whilst simulating risk assesment with Python
\item Traded indices, commodities and forex pairs using leverage and margin with a risk management strategy 
\item Engineered a trading algorithm using Python and Quantitave Analysis for automated trading and backtesting
\end{itemize}
\tightdivider

\cvevent{IT administrator}{Mesopotamia Group}{Jun 2020 - Mar 2021 (10 Months)}{Amman Jordan}
\begin{itemize}
\item Collaborated with a team to build a local file server on Ubuntu built using Apache and Nextcloud
\item Administered the server using Bash, Python and Nmap with SSH tunneling and port forwarding
\item Engineered a load balancer using MPI and C to distribute traffic between servers increasing availability by 34\%
\item Analyzed the network data logs using Python to predict and prevent cyber attacks
\item Implemented Linux networking and protocols using IP addressing and routing for cloud hosting
\end{itemize}

% \switchcolumn
\medskip

\cvsection{ \faBook \, Education}
\cveducation{BEng (Hons) in Electronics and Computer}{University of Leeds (UoL)}{Sept 2019 - June 2024}{First-Class Honors}
\begin{itemize} 
\item Embedded systems, Microprocessors, FPGA, Circuit Analysis, Signal processing, Networking,  Compiler design
\item Cloud and Parallel computing, Secure and Distributed systems, Cybersecurity, Software Engineering, Research
\end{itemize}

% Engineering and Discrete Mathematics | Cloud and Parallel computing | Electronics and Circuit analysis | Communications and Signals | Networking and Cybersecurity | User interface and Compiler Design | Microprocessors and Embedded systems | Autodesk AutoCAD | Advanced Physics and Math
% \end{itemize}
% \tightdivider

% \cveducation{Foundation in Engineering and Computing}{\href{https://www.leedsisc.com/our-courses/international-foundation-year/science-engineering-computing}{University of Leeds (LISC)}}{Sept 2020 -- June 2020}{93\%}

% \begin{itemize}
% \item Advanced Physics and Math
% \item Autodesk AutoCAD
% \end{itemize}
% \tightdivider
% \cvevent{IB Science certificate}{\href{http://cambridge.edu.jo/}{Cambridge High School}}{Sept 2017 -- June 2019 \quad \faCheckSquare \ \ 40/42}{}

% \begin{itemize}
% \item Physics and Math
% \item Languages and Economics
% \end{itemize}
\medskip

% //TODO: add skills required by emplooyers not stupid arm keil or dome shit 
\cvsection{\faUser \, Skills}
\textbf{Technical}: 
\cvtag{C \textbackslash \ C++}
\cvtag{Python}
\cvtag{Java}
\cvtag{C\#}
\cvtag{JavaScript} 
% \cvtag{HTML5}
% \cvtag{CSS} 
% \cvtag{Typescript} 
% \cvtag{Dart} 
% \cvtag{Matlab}
% \cvtag{REST API}
% \cvtag{CI \textbackslash \ CD}
% \cvtag{DevOps} 
% \cvtag{SQL} 
% \cvtag{Bash} 
% \cvtag{PowerShell} 
% \cvtag{Verilog}
% \cvtag{Assembly} 
% \cvtag{Latex}  
\cvtag{Linux}
\cvtag{Docker}
\cvtag{Azure}
\cvtag{AWS}
\cvtag{Jira}
\cvtag{Git}
\cvtag{Jenkins}
\cvtag{Terraforms}
\cvtag{Kubernetes}
% \cvtag{IBM Cloud}
% \cvtag{Firebase}
% \cvtag{GCP}
% \cvtag{CI/CD}
\cvtag{SQLite}
% \cvtag{IBM Db2}
\cvtag{MongoDB}
% \cvtag{Postgres}
% \cvtag{.NET}
\cvtag{.NET WPF}
\cvtag{React Js}
\cvtag{QT Framework}
\cvtag{Flutter}
% \cvtag{Next.js}
% \cvtag{Tailwind CSS}
% \cvtag{Material UI}
% \cvtag{Chakra UI}
% \cvtag{Qt}
% \cvtag{Django}
% \cvtag{Flask}
% \cvtag{Express.js}
% \cvtag{Springboot}
% \cvtag{ASP.NET}
\cvtag{NumPy}
% \cvtag{Keras}
\cvtag{KerasRL}
\cvtag{Matplotlib}
% \cvtag{Pytorch}
\cvtag{SciPy}
\cvtag{TensorFlow}
\cvtag{Pandas}
% \cvtag{Scikit-learn}
\cvtag{OpenCv}
\cvtag{CMake}
% \cvtag{Make}
% \cvtag{Maven}
\cvtag{Node.js}
\cvtag{OpenCL}
\cvtag{OpenGL}
\cvtag{OpenMP}
\cvtag{CUDA}
% \cvtag{Logism}
% \cvtag {Office Suite} 
\cvtag{Arm Mbed Os}
\cvtag{Arm Keil}
% \cvtag{FPGA} 
% \cvtag{Simplicity Studio}
% \cvtag{Rasberry Pie}
\cvtag{Arduino} 
% \cvtag{Multism}
% \cvtag{Intel Quartus}
\cvtag{Blender}
\cvtag{Photshop}
\cvtag{AutoCAD} \\
\tightdivider \\
\textbf{Soft}: 
\cvtag{Management}
\cvtag{Leadership}
% \cvtag{Hard-working}
\cvtag{Motivated}
\cvtag{Deterministic}
\cvtag{Communication}
\cvtag{Problem solver} \\
% \vspace{2.5mm}
\tightdivider \\

% \cvsection{\faLanguage \, Languages}
\textbf{Languages}:  \textbf{Arabic} (Full Professional) \hfill \textbf{English} (Full Professional) \hfill \textbf{Spanish} (Elementary)

%% spacing consistent... sorry. Use \smallskip, \medskip,
%% \bigskip, \vpsace etc to make ajustments.
% \medskip

\cvsection{\faTasks \, Projects}

\cvevent{JPS}{}{Oct 2022 - Jul 2023 \quad \faMapPin \ \ JCI}{}
\begin{itemize}
\item Re-designed a legacy CLI project to a GUI with C\# .NET WPF using MVVM and multithreaded design
\item Invented an object-oriented scripting language with a custom compiler and runtime memory access.
\item Integrated SQLite database, Excel I/O features, data analysis tools and serial communication protocols 
\item Communicated with engineers to develop and manage a product life cycle on the app's design and features
\end{itemize}

\tightdivider

\cvevent{MCPP}{}{Apr 2021 - Oct 2022 \quad \faMapPin \ \ Personal}{}
\begin{itemize}
\item Created a cross-platform math library for C++ used for matrix algebra, statistics and machine learning.
\item Engineered with CI/CD using Jenkins, unit tests, Docker, Cmake, Doxygen and Git for version control
\item Improved performance by 40\% using parallel computing with OpenMP, OpenCL and CUDA
\end{itemize}

\cvevent{Jobhive}{}{Sept 2023 - Ongoing \quad \faMapPin \ \ Personal}{}
\begin{itemize}
\item Developed a fullstack web app with Nextjs, Reactjs, Chakra-ui and Typescript managing frontend and backend.
\item Engineered a monolothic backend REST API routing and endpoints with CRUD Operations using nextjs 
\item Collaborated with a team using Jira issues and git to sustain clean, maintainable and efficient code
\item Developed project management skills from designing the app in figma to implementing it 
\end{itemize}

\tightdivider


\cvevent{NotePad}{}{Oct 2022 - Jan 2023 \quad \faMapPin \ \ Personal}{}
\begin{itemize}
\item Programmed a cross-platform note app with Flutter and Dart
\item Established backend using cloud authentication services with Firebase and storage with SQL CRUD 
\item Using the BLoC pattern for state management and the Provider package for dependency injection
\item Used KerasRL library for reinforcement learning with human feedback to predict user's next note
\end{itemize}
\tightdivider

\cvevent{Home Automation Embedded System}{}{Jan 2021 - May 2021 \quad \faMapPin \ \ UoL}{}
\begin{itemize}
\item Engineered circuits for I/O functionality using data sensors, potentiometers, buttons and resistors.
\item Programmed control system using C++ for home appliances such as temperature, air conditioning and lighting 
\item Developed menu oriented user interface using LCD display and joystick deployed on STM32 board ARM MCU
\end{itemize}

\cvevent{TBot: Machine learning Trading Bot}{}{Apr 2021 - Oct 2021 \quad \faMapPin \ \ UoL}{}
\begin{itemize}
\item Developed gradient boosting machine learning model using Python and Scikit-learn for stock prediction
\item Anlayzed and processed data using Pandas, Numpy and Scipy for training and testing the model 
\item Backtested the model using historical data and simulated trading using Python and Matplotlib
\item Developed a REST API and a user interface to interact with the bot using Flask and PyQT
\end{itemize}

\cvevent{Vega: Chat web app for IoT Circuits}{}{Oct 2023 - Jun 2024 \quad \faMapPin \ \ UoL}{}
\begin{itemize}
\item Developed a chat bot web application with Typescript, React.js, Next.js, Reddis and Tailwind CSS
\item Programmed natural language processing algorithms using LLMs to interpet and execute user messages 
\item Developed a Python Flask REST Server with SQLite on the Raspberry Pi to communicate with the chat bot
\item Engineered a circuit with sensors, fans, LCD and buttons to execute the user's commands from the chat bot 
\end{itemize}

\medskip

\cvsection{\faCertificate \, Professional Courses}
\cvevent{\href{https://coursera.org/share/04cc7aedb131d2ec5e627c5b6943365b}{Data Analysis with Python}}{}{Oct 2021}{}
\begin{itemize}
\item Develop Python code for cleaning and preparing data for analysis 
\item Perform exploratory data analysis and analytical techniques to datasets using Pandas, Numpy and Scipy
\item Manipulate data using dataframes, understand data distribution, perform correlation and create data pipelines
\item Build and evaluate regression models using machine learning scikit-learn for prediction and decision making
\end{itemize}
\tightdivider

\cvevent{\href{https://coursera.org/share/e3d5bbb39e72fa4deb5730f2d48c8979}{Databases and SQL for Data Science with Python}}{}{Sept 2021}{}
\begin{itemize}
\item Analyze data within a relational database on the cloud using IBM DB2, SQL, Python
\item Construct SQL CRUD operations and compose views, transactions, stored procedures and joins.
\end{itemize}
\tightdivider

\cvevent{\href{https://coursera.org/share/6878977db8f2fbc30ac46f61dced356a}{Python for Data Science, AI \& Development}}{}{Aug 2021}{}
\begin{itemize}
\item Demonstrate proficiency in using Python libraries such as Pandas, Numpy, and Beautiful Soup
\item Access web data using APIs and web scraping from Python in Jupyter Notebooks.
\end{itemize}
\tightdivider

% \cvevent{\href{https://coursera.org/share/d8a5a903819c5c1630966c8f31e10d2f}{Data Science Methodology}}{}{Aug 2021}{}
% \begin{itemize}
% \item Apply the six stages in the Cross-Industry Process for Data Mining (CRISP-DM) to analyze a case study.
% \item Evaluate which analytic model is appropriate among predictive, descriptive, and classification models used.
% \end{itemize}
% \tightdivider
\medskip


% \cvevent{\href{https://coursera.org/share/93d4419e7af771becc207723b671843a}{Tools for Data Science}}{}{Jul 2021}{}
% \begin{itemize}
% \item Utilize R, Python using RStudio and Jupyter Studio
% \item IBM tools such as IBM Watson Studio
% \end{itemize}
% \tightdivider

% \cvevent{\href{https://coursera.org/share/a785833336cd1025863f958a616159ca}{What is Data Science?}}{}{Jun 2021}{}
% \begin{itemize}
% \item Introduction to data science and big data
% \item Structure of data science reports
% \end{itemize}
% \tightdivider

% \cvevent{\href{https://www.coursera.org/account/accomplishments/verify/93PY64WDHCXY?utm_source=ios&utm_medium=certificate&utm_content=cert_image&utm_campaign=sharing_cta&utm_product=course}{Matrix Algebra for Engineers}}{}{Nov 2020}{}
% \begin{itemize}
% \item Basics of matrix algebra and properties of matrices
% \item Linear algebra and its use in statistical analysis
% \end{itemize}


\cvsection{\faUsers \, Extracurricular}
\cvevent{Event Coordinator Volunteer}{IEE RAS ICRA}{Nov 2021 - Feb 2022}{London, UK} \\
\vspace{-2.8mm}
\tightdivider \\
\cvevent{Web Developer hackathon}{Royal Hackaway v6}{Nov 2021 - Feb 2022}{London, UK} \\
\vspace{-2.8mm}
\tightdivider \\
\cvevent{Game Developer hackathon}{Pygames hackathon Microsoft}{Nov 2021 - Feb 2022}{Remote} \\
\vspace{-2.8mm}
\tightdivider \\
\cvevent{AI Agent hackathon}{Lablab}{Nov 2021 - Feb 2022}{Remote} 
% \vspace{-2.8mm}
% \tightdivider \\
% \cvevent{AI Agent hackathon}{Lablab}{Nov 2021 -- Feb 2022}{Remote} 
% \cvsection{\faBookmark \, Publications}

% \cvsection{\faBookmark \, References}

% \cvsection{\faBookmark \, Awards}

















% \divider

% \cvevent{MEng\ in Artficial Intelligence}{University of Southampton}{Sept 2023 -- June 2024}{}

% \divider

% \cvsection{Referees}

% % \cvref{name}{email}{mailing address}
% \cvref{Prof.\ Alpha Beta}{Institute}{a.beta@university.edu}
% {Address Line 1\\Address line 2}

% \divider

% \cvref{Prof.\ Gamma Delta}{Institute}{g.delta@university.edu}
% {Address Line 1\\Address line 2}


% \switchcolumn
% \vspace{100mm}
% \cvsection{Publications}

% \nocite{*}

% \printbibliography[heading=pubtype,title={\printinfo{\faBook}{Books}},type=book]

% \divider

% \printbibliography[heading=pubtype,title={\printinfo{\faFileTextO}{Journal Articles}},type=article]

% \divider

% \printbibliography[heading=pubtype,title={\printinfo{\faGroup}{Conference Proceedings}},type=inproceedings]

% %% Switch to the right column. This will now automatically move to the second
% %% page if the content is too long.

\end{paracol}


\end{document}
